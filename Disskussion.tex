\documentclass[SRC.tex]{subfiles}

\begin{document}
	Da verdensbefolkningen er i markant vækst, kræves der flere ressourcer, end vi anvender nu. Dette kan blive problematisk, da eksempelvis 81 procent af USA’s elproduktion, som tidligere nævnt, bygger på fossile brændsler. Dermed vurderer forskere, at de naturlige ressourcer er opbrugt om 53 år. Derfor afhænger verdens eksistens og udvikling af, at det traditionelle ressourceforbrug særligt til produktion nytænkes og omlægges til grøn energi baseret på vedvarende energikilder som b.la. sol- og vindenergi. 
	
	Ud fra et miljømæssigt og vedvarende perspektiv virker EuroDishen teoretisk som et godt bud på et grønnere alternativ, da den opererer på solenergi, som opvarmer Stirling motoren, og dermed omdanner solens energi til anvendelig el. Ligeledes viser beregning \eqref{eq:nyt}, at Stirling motorens nyttevirkning, udregnet i den teoretisk opstillede model, er \SI{58}{\percent}. Dermed har den idealiserede Stirling motor teoretisk en bedre nyttevirkning og omdannelsesevne end en Mono-krystallinsk solcelle, hvis nyttevirkning er på 22-\SI{27}{\percent}.  
	
	Selvom Stirling motoren på papiret virker som et oplagt, grønt og godt alternativ, er den opstillede model og den idealiserede Stirling motor langt fra realistisk, da en EureDish har en overordnet nyttevirkning på \SI{22}{\percent}, hvilket er omkring halvdelen af den opstillede model (Schiel, W., og Laing). En af grundende til, at den opstillede model afviger så meget fra det reelle system, er fordi der er blevet opstillet en teoretisk og idealiseret model i form af en kredsproces, der består af termodynamiske processor, som alle tager afsæt i idealgasligningen. Ligeledes tages der ikke højde for gassens afgivelse af varme til omgivelserne, eller rettere sagt til materialet, som indespærrer gassen. En anden faktor, som modellen ikke tager højde for i dens opstilling af nyttevirkningen for solenergi til el, er den strøm og energi, som systemet i sig selv bruger. Selve Eurodishen kan rotere omkring sig selv, netop for at kunne udnytte solens indstråling optimalt. For at den kan opnå dette, så sidder der en sensor og en styrecomputer, som behandler sensordataen og udregner, hvor den skal dreje hen. Ligeledes anvendes der også strøm og energi til at afkøle gassen i Stirling motoren. Dette er især vigtigt, da det er temperaturforskellen mellem de to køleflader, som er med til at bestemme nyttevirkningen for maskinen. Da den opstillede model ikke er repræsentativ for virkeligheden, så giver den ovenstående sammenligning mellem modellen og et solpanel heller ikke et realistisk resultat. Her viser det sig nu, at solcellen er mere nyttig end EuroDishen samt, at solpanelet ikke skal tilføres noget energi i drift, ligesom EuroDishen skal.  
	
	For at kunne have opstillet en mere realistisk model, så skulle der have været mere data inden over, hvorved der kunne tages højde for flere forskellige faktorer. Andre faktorer, som kan have en indflydelse på nyttevirkningen af en EuroDish kunne være geografisk placering. Altså i forhold til klima og deraf mængden af soltimer og indfaldsvinklen af solen. Som det fremgår på figur \ref{fig:world-solar-isolation-map}, så er Danmark og generelt Norden et uoptimalt sted for udnyttelsen af en EuroDish og generelt for udnyttelsen af solenergi. I stedet kunne Stirling motoren måske være et godt og reelt alternativ, hvis den udnyttes og anvendes i de røde lande og kontinenter fra figur \ref{fig:world-solar-isolation-map}, som Afrika, Sydamerika, Australien osv. Kontinenternes solrige klima vil biddrage til bedre forudsætninger for Stirling motoren, da den bygger på solenergi. 
	
	\begin{figure}[h!]
		\centering
		\includegraphics[width=0.7\linewidth]{"Billeder/World solar isolation map"}
		\caption{Verdenskort med anførte soltimer. Kilde: \url{https://rameznaam.com/wp-content/uploads/2013/11/World-Insolation-Map.gif}}
		\label{fig:world-solar-isolation-map}
		%kilde: https://1.bp.blogspot.com/-_e1as8fADig/ThWJcC7Bh5I/AAAAAAAAA2Y/BGBFXggY_C8/s1600/World+solar+isolation+map.jpg
	\end{figure}
	Anskues EuroDishen ud fra den teoretisk opstillede model, som fremstiller en idealiseret Stirling motor, så fremstår den som et godt alternativ til anvendelsen af det traditionelle ressourceforbrug, da nyttevirkningen er \SI{58}{\percent}. Anskues EuroDishen i stedet ud fra et realistisk synspunkt, er den opstillede model ikke et realistisk bud på et grønnere alternativ, da den reelle nyttevirkning er \SI{22}{\percent} mens der ligeledes ikke er taget højde for, at motoren også anvender strøm i omdannelsesprocecssen, samt de geografiske faktorer. For at man kan anvende en model som forklaring på virkelige fænomener, skal der være en teoretisk og praktisk overensstemmelse. En overensstemmelse som ikke er tilstede i sammenligningen af resultaterne fra den idealiserede motor og den reelle motor. Dermed kan modellen ikke anvendes som et realistisk bud på et grønnere alternativ, da den afviger for meget fra den reelle EuroDish, dog fungerer den reelle EuroDish som en god grøn energiproduktion, da den primært kører på den vedvarende solenergi. 
	
	
\end{document}