\documentclass[SRC.tex]{subfiles}

\begin{document}
	Den opstillede model er langt fra realistisk, da en EureDishen har en overordnet nyttevirkning 
	på \SI{22}{\percent}, hvilket er omkring halvdelen af den opstillede model. %kilde: file:///tmp/mozilla_ckl0/ISEC_10-Osnabrck_Paper_Schiel_Laing.pdf
	En af grundende til, at den opstillede model afviger, så meget fra det reele system, er fordi
	der er blevet opstillet en teoretisk og idealiseret model i form af en kredsproces, der består af termodynamiske proecssor, som alle tager afsæt i idealgasligningen. Ligeledes tages der ikke højde
	for gassen afgivelse af varme til omgivelserne, eller rettere sagt til materialet, som indespærer gassen. En anden faktorer, som modellen ikke tager højde for i dens opstilling af nyttevirkningen for solenergi til el, er den strøm og energi, som systemet i sig selv bruger. Selve Eurodishens kan rotere omkring sig selv, netop for at kunne udnytte solens indstråling optimalt. For at den kan opnå dette, så sidder der en sensor og en styrecomputer, som behandler sensordataen og udregner, hvor den skal dreje hen. Ligeledes anvendes der også strøm og energi til at afkøle gassen i Stirling motoren. Dette er især vigtigt, da det temperaturforskellen, mellem de to køleflader, som er med til at bestemme nyttevirkningen for maskinen. Da den opstillede model ikke er repræsentativ for virkeligheden, så giver den ovenstående sammenlignig mellem modellen og et solpanel heller ikke et realistisk resultat. Her viser det sig nu, at solcellen er mere nyttig end EuroDishen samt, at solpanelet ikke skal tilføres noget energi i drift ligesom EuroDishen skal. 
	
	For at kunne have opstillet en mere realistisk model, så skulle der have været mere data indenover, hvorved der kunne tages højde for flere forskellige faktorere. Andre faktorer, som kan have en indflydelse på nyttevirkningen af en EuroDish kunne være geografisk placering. Altså i forhold til klima og deraf mængden af soltimer og indfaldsvinklen af solen. Som det fremgår på figur \ref{fig:world-solar-isolation-map}, så er Danmark og generelt norden et uoptimalt sted for udnyttelsen af en EuroDish og generelt for udnyttelsen af solenergi. 
	
	\begin{figure}[h!]
		\centering
		\includegraphics[width=0.7\linewidth]{"Billeder/World solar isolation map"}
		\caption{Verdenskort med anførte soltimer. Kilde: \url{https://rameznaam.com/wp-content/uploads/2013/11/World-Insolation-Map.gif}}
		\label{fig:world-solar-isolation-map}
		%kilde: https://1.bp.blogspot.com/-_e1as8fADig/ThWJcC7Bh5I/AAAAAAAAA2Y/BGBFXggY_C8/s1600/World+solar+isolation+map.jpg
	\end{figure}
	
\end{document}