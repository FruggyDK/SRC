\documentclass[SRC.tex]{subfiles}

\begin{document}
	På baggrund af de termodynamiske processor samt kredsprocesser, der beskrives i redegørelsen, opstilles der i analysen ved hjælp af metoden, matematisk modellering, en idealiserede model af en EuroDish. Beregninger af den opstillede model resulterede i en teoretisk nyttevirkning på  \SI{58}{\percent}. Sammenlignet med et solpanel bestående af Mono-krystallinske solceller fremstår den idealiserede model som værende ca. dobbelt så effektiv, da nyttevirkningen for solpanelet er 22-\SI{27}{\percent}. Det diskuteres frem, at analysens resultater ikke er realistiske eller repræsentative, da der ikke tages højde for nogle realistiske faktorer, såsom klima, alle afgivelser af varme til omgivelserne, at der i omdanningsprocessen anvendes strøm samt at den reelle nyttevirkningen for en EuroDish er \SI{22}{\percent}. Dermed kan det konkluderes, at modellen ikke kan anvendes som et realistisk bud på et grønnere alternativ, da den afviger for meget fra den reelle EuroDish, dog fungerer den reelle EuroDish som en god grøn energiproduktion, da den primært kører på den vedvarende solenergi. 
\end{document}