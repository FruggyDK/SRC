\documentclass[SRC.tex]{subfiles}

\begin{document}
	
	For at kunne regne på, hvor effektiv en EuroDish er, så er det nødvendigt at opstille
	en matematisk model, som der kan regnes på. En sådan model, tager afsæt i selve den del
	på EuroDishen, som omdanner solens energi til mekanisk energi, altså Stirlingmotoren. 
	
	\subsection{Analyse af en idealiseret Stirlingmotor}
	En idealiseret stirling motor er en model, som kan opstilles til en 
	Stirling motor, hvor der termodynamiske processor sammensættes til en kredsproces. Da 
	de termodynamiske processor er udledt af idealgaslignigen, så giver det 
	god mening, at en model opstillet med disse også vil være ideal. Den idealeliseret Stirling motor vil derfor ikke matche en realistisk Stirling Motor. Den idealiserede stirling motors \(pV\)-diagram kan ses på figur \ref{fig:stirlingcycle}
	
	\begin{figure}[h!]
		\centering
		\includegraphics[width=0.4\linewidth]{Billeder/330px-Stirling_Cycle_color.svg}
		\caption{En idealiseret Stirling motos kredsproces indtegnet i et \(pV\)-diagram. Kilde}
		%kilde: \url{https://en.wikipedia.org/wiki/Stirling_engine#/media/File:Stirling_Cycle_color.svg}
		\label{fig:stirlingcycle}
	\end{figure}
	

	Hvor proces:
	\begin{enumerate}[]
		\item \quad \(1 \rightarrow 2\): er en isoterm ekspansion
		\item \quad \(2 \rightarrow 3\): er en iskor afkøling
		\item \quad \(3 \rightarrow 4\): er en isoterm kompression
		\item \quad \(4 \rightarrow 1\): er en isokor opvarmning
	\end{enumerate}
	
	
	\subsubsection{Arbejde udført af en idealiseret Stirlingmotor}
	Nettoarbejdet for en idealiseret Stirling motor kan beregnes ved at finde arealet, der afgrænses
	af de termodynamiske processor  i \(pV\)-diagrammet. Der udføreres kun et arbejde ved 
	proces \(1\rightarrow2\), altså den isoterme ekspansion og ved den proces \(3 \rightarrow 4\), 
	den isoterme kompression. Arbejdet for en isoterm proces kan opfattes, som araelet under den 
	afbillede graf i diagrammet. Derfor må nettoarbejdet være lig med forskellen på den øverste
	process og den nederste process, som rent matematisk kan udregnes som
	\begin{equation}
		A_{\text{netto}} = \int_{V_1}^{V_2} p \, dV - \int_{V_3}^{V_4} p \, dV 
		\label{eq:1}
	\end{equation}
	Selve arbejdet, som en isoterm udfører undervejs, kan findes ved at betrage ligning \eqref{eq:1},
	idealgasligningen, og huske at for en isoterm proces er temperaturen \(T\) konstant, samt stofmængden \(n\), da det er en kredsproces. Ud fra dette kan et udtryk for det udførte arbejde skrives som
	\begin{subequations}
		\begin{align}
		\int_{V_A}^{V_B} p \, dV &= \int_{V_A}^{V_B} \frac{nRT}{V} \, dV  = nRT \cdot 	\int_{V_A}^{V_B} \frac{1}{V}\, dV \\
		&= nRT \cdot \left[\ln V\right]_{V_A}^{V_B}  = nRT \cdot (\ln V_\text{B} - \ln V_\text{A})\\
		&=nRT \ln\left(\frac{V_B}{V_A}\right)
		\label{eq:workint}
		\end{align}
	\end{subequations}
	hvor \(V_{\text{B}}\) og \(V_{\text{B}}\) er henholdsvis slut- og startvolumenet. Fra trin b til c
	i ligning \eqref{eq:workint} anvendes, at \[\ln(A)-\ln(B) = \ln\left(\frac{A}{B}\right). \]
	Dette er den samme ligning, som blev introduceret i redegørelse, navnligt ligning \eqref{eq:10},
	den eneste forskel er fortegnet. I redegørelsen var det arbejdet, som omgivelserne skulle udføre
	på gassen, hvorimod det her er gassens arbejde. 
	Ved at substituere ligning \eqref{eq:workint} i ligning \eqref{eq:1}, så kan integralet for 
	nettoarbejdet evalueres på følgende vis
	\begin{equation}
		A_{\text{netto}} = nRT_H\ln\left(\frac{V_2}{V_1}\right)-nRT_L\ln\left(\frac{V_3}{V_4}\right)
	\end{equation}
	hvor subskriftene \(H\) og \(L\) denoterer henholdsvis den høje og lave temperatur i den isoterme
	proces. Ved at obsevere \(pV\)-diagrammet (indsæt figur), så kan det ses at \(V_4 = V_1\) samt 
	at \(V_3 = V_2\), netop da to lodrette processor er isokore processor, hvor volumenet holdes konstant. Herved kan ligningen omskrives til
	\begin{equation}
	A_{\text{netto}} = nRT_H\ln\left(\frac{V_2}{V_1}\right)-nRT_L\ln\left(\frac{V_2}{V_1}\right)
	\end{equation}
	Ved at faktorisere kan udtrykket skrives som
	\begin{equation}
		A_{\text{netto}}= nR\ln\left(\frac{V_2}{V_1}\right)\cdot (T_H - T_L)
		\label{eq:arbejde}
	\end{equation}
	Så arbejdet, som udføres på omgivelserne, af en idealiseret Stirling motor kan beregnes ved anvendelse af ligning \eqref{eq:arbejde}
	kilde \footnote{pdf, plus bog}
	
	\subsubsection{Tilførsel af varme til systemet}
	Varmen, \(Q\), som skal tilføres til gassen i kredsprocessen kan udledes ud fra ligning
	\begin{equation}
		Q_{\text{tilført}} = nRT_H\ln\left(\frac{V_2}{V_1}\right)
		\label{eq:varme}
	\end{equation}
	Skriv mere here. Når den ideliserede cyklus når proces \(1 \rightarrow 2\), så sker der en 
	ekspansion, altså volumenet på gassen bliver større. Derfor skal der tilføres en hvis mængde 
	energi til gassen for at holde temperaturen i gassen konstant. 
	\subsubsection{nyttevirkning af en idealiseret Stirling motor}
	For en kraftvarmemaskine, som Stirling maskinen er dens nyttevirkning defineret som forholdet mellem maskinens arbejde og den tilførte energi. Nytte virkningen kan beregnes ved 
	\begin{equation}
		\eta = \frac{A_{\text{maskine}}}{Q_{\text{tilført}}}
		\label{eq:nyttevirkning}
	\end{equation}
	hvor \(A_{\text{maskine}}\) er maskinens arbejde og \(Q_{\text{tilført}}\) e varmen, der tilføres maskinen undervejs. For en idealiseret Stirling motor, så kan nyttevirkningen udregnes ved at indsææte ligning \eqref{eq:arbejde} og \eqref{eq:varme} i ligning \eqref{eq:nyttevirkning}, som resulterer i
	\begin{equation}
		\eta = \frac{nR\ln\left(\frac{V_2}{V_1}\right)\cdot (T_H - T_L)}{nRT_H\ln\left(\frac{V_2}{V_1}\right)} 
	\end{equation}
	Her kan det let ses, at \(nR\ln\left(\frac{V_2}{V_1}\right)\) i tæller og nævner vil udligne hinanden og så kan nyttevirkningen nu beregnes med
	\begin{equation}
		\eta = \frac{T_H-T_L}{T_H}
		\label{eq:nytte}
	\end{equation}
	Heraf kan det ses, at nyttevirkningen af en idealiseret stirling motor udelukkende afhænger af den højeste og laveste temperatur. Ved at omskrive ligning \eqref{eq:nytte} som følgende
	\begin{subequations}
		\begin{align}
			\eta &= \frac{T_H-T_L}{T_H} \\
			 	 &= \frac{T_H}{T_H}-\frac{T_L}{T_H} \\
			 	 &= 1 -\frac{T_L}{T_H}
		\end{align}
	\end{subequations}
	så kan det obseveres, at nyttevirkningen for en idealiseret stirling motor aldrig kan overstige 1. 
	\subsection{Beregning på modellen}
	Som det fremstod i det forrige, så kan nyttevirkningen for en idealiseret Stirling motor beregnes, hvis man kender den maksimale og minimale temperatur for gassen i kredsprocessen. I EuroDishens anvendes en Stirling motoren SOLO Stirling 161, der produceres af SOLO Kleinmotoren. I databladet for den anvendte Stirling motor fremgår det at den maksimale driftstemperatur er \SI{650}{\celsius}, dog så fremstår den minimale temperatur i databladet. Derfor tages der i det følgende afsæt i en anden Stirling motor, som har en maksimal temperatur på \SI{1054}{\kelvin} og en minimum på \SI{308}{\kelvin}. %kilde: file:///tmp/mozilla_ckl0/ThermalmodelEurodishJSEE.pdf
	Ud fra disse data, så kan den teoretiske og idealiserede nyttevirkning for denne Stirling motor beregnes med ligning \eqref{eq:nytte}, som resulterer i
	\begin{equation}
		\eta_{\text{Stirling}} = \frac{\SI{1054}{\kelvin} - \SI{308}{\kelvin}}{\SI{1054}{\kelvin}} = 0.708 = \SI{70.8}{\percent}
	\end{equation}
	Det vil altså sige, at for al energi som tilføres til den idealiserede Stirling maskine, så vil en andel på \(\SI{70.8}{\percent}\) blive omdannet til mekanisk energi. For at EuroDishen kan generer el, så er der koblet en elgenereator på Stirling motoren, som omdanner det mekaniske energi til elektricitet. I det tidligere nævnte datablad, så fremgår virkningsgraden for den anvendte elgenerator og er opgivet til at være \(\eta_{\text{gen}} = \SI{94}{\percent}\). Heraf kan den nyttevirkning fra tilført varmeenergi til elektrisk energi, generede af generatoren, findes ved produktet af disse nyttevirkninger
	\begin{equation}
		\eta_{\text{q}\rightarrow\text{el}} =  \eta_{\text{stirling}} \cdot \eta_{\text{gen}} = 0.708 \cdot 0.94 = 0.665 = \SI{66.5}{\percent}
	\end{equation}
	Så for den energi, som tilføres til Stirling generatoren via sollys, så vil en andel af \SI{67}{\percent} blive omdannet til elektrisk energi, der kan sendes ud til elnettet. 
	\subsection{Sammenligning}
\end{document}