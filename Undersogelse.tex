\documentclass[SRC.tex]{subfiles}

\begin{document}
	
	For at kunne regne på, hvor effektiv en EuroDish er, så er det nødvendigt at opstille
	en matematisk model, som der kan regnes på. En sådan model, tager afsæt i selve den del
	på EuroDishen, som omdanner solens energi til mekanisk energi, altså Stirlingmotoren. 
	
	\subsection{Analyse af en idealiseret Stirlingmotor}
	En idealiseret stirling motor er en modelering af....
	
	\subsubsection{Arbejde udført af en idealiseret Stirlingmotor}
	Nettoarbejdet for en idealiseret Stirling motor kan beregnes ved at finde arealet, der afgrænses
	af de termodynamiske processor  i \(pV\)-diagrammet. Der udføreres kun et arbejde ved 
	proces \(1\rightarrow2\), altså den isoterme ekspansion og ved den proces \(3 \rightarrow 4\), 
	den isoterme kompression. Arbejdet for en isoterm proces kan opfattes, som araelet under den 
	afbillede graf i diagrammet. Derfor må nettoarbejdet være lig med forskellen på den øverste
	process og den nederste process, som rent matematisk kan udregnes som
	\begin{equation}
		A_{\text{netto}} = \int_{V_1}^{V_2} p \, dV - \int_{V_3}^{V_4} p \, dV 
		\label{eq:1}
	\end{equation}
	Selve arbejdet, som en isoterm udfører undervejs, kan findes ved at betrage formel \eqref{eq:1},
	idealgasligningen, og huske at for en isoterm proces er temperaturen \(T\) konstant, samt stofmængden \(n\), da det er en kredsproces. Ud fra dette kan et udtryk for det udførte arbejde skrives som
	\begin{subequations}
		\begin{align}
		\int_{V_A}^{V_B} p \, dV &= \int_{V_A}^{V_B} \frac{nRT}{V} \, dV  = nRT \cdot 	\int_{V_A}^{V_B} \frac{1}{V}\, dV \\
		&= nRT \cdot \left[\ln V\right]_{V_A}^{V_B}  = nRT \cdot (\ln V_\text{B} - \ln V_\text{A})\\
		&=nRT \ln\left(\frac{V_B}{V_A}\right)
		\label{eq:workint}
		\end{align}
	\end{subequations}
	hvor \(V_{\text{B}}\) og \(V_{\text{B}}\) er henholdsvis slut- og startvolumenet. Fra trin b til c
	i formel \eqref{eq:workint} anvendes, at \[\ln(A)-\ln(B) = \ln\left(\frac{A}{B}\right). \]
	Ved at substituere formel \eqref{eq:workint} i formel \eqref{eq:1}, så kan integralet for 
	nettoarbejdet evalueres på følgende vis
	\begin{equation}
		A_{\text{netto}} = nRT_H\ln\left(\frac{V_2}{V_1}\right)-nRT_L\ln\left(\frac{V_3}{V_4}\right)
	\end{equation}
	hvor subskriftene \(H\) og \(L\) denoterer henholdsvis den høje og lave temperatur i den isoterme
	proces. Ved at obsevere \(pV\)-diagrammet (indsæt figur), så kan det ses at \(V_4 = V_1\) samt 
	at \(V_3 = V_2\), netop da to lodrette processor er isokore processor, hvor volumenet holdes konstant. Herved kan ligningen omskrives til
	\begin{equation}
	A_{\text{netto}} = nRT_H\ln\left(\frac{V_2}{V_1}\right)-nRT_L\ln\left(\frac{V_2}{V_1}\right)
	\end{equation}
	Ved at faktorisere kan udtrykket skrives som
	\begin{equation}
		A_{\text{netto}}= nR\ln\left(\frac{V_2}{V_1}\right)\cdot (T_H - T_L)
		\label{eq:arbejde}
	\end{equation}
	Så arbejdet, som udføres på omgivelserne, af en idealiseret Stirling motor kan beregnes ved anvendelse af formel \eqref{eq:arbejde}
	kilde \footnote{pdf, plus bog}
	\subsection{Beregning på modellen}
	\subsection{Sammenligning}
\end{document}