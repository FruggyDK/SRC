\documentclass[SRC.tex]{subfiles}

\begin{document}
	Igennem tidens løb har mennesket adapteret sig til og anvendt sine omgivelser til deres fordel, ved at udarbejde redskaber, som gjorde dagligdagens opgaver nemmere. Denne tilpasning har fundet sted og finder stadig sted den dag i dag. Før i tiden har udviklingen været motiveret ved at skabe hurtigere og mere effektive redskaber og maskineri, som kunne erstatte arbejdernes jobs. Denne forskydning, hvor mennesker hele tiden har fundet nyere, smartere og mere effektive måder at udføre jobs på, har medført at flere mennesker kan arbejde i det tertiære og sekundære erhverv. Ultimativt vil denne udvikling nok resultere i en næsten komplet overtagelse af arbejde. I takt med at jordens befolkningsmængde stiger eksponentielt, så kræves der flere og flere ressourcer, både i form af energi, varme og fødevarer, mere specifikt så har verdensenergiforbruget mere end tredoblet siden 1970 \footnote{\url{https://www.drivkraftdanmark.dk/wp-content/uploads/2019/05/DD_Energistatistik_2019_WEB-spreads.pdf}}. Produktionen af el i USA kommer primært fra fossile brændsler, som er en andel på 81 procent, hvoraf 61.8 procent kommer fra naturgas og kul\footnote{\url{https://www.agfoundation.org/common-questions/view/Where-does-energy-come-from}}. For at kunne følge med på denne trend, med en voksende befolkningsmængde og et voksende energiforbrug, så vurdere forskere, at de naturlige ressourcer, som kul og olie, vil blive opbrugt indenfor ca. 53 år \footnote{\url{https://eu.usatoday.com/story/money/business/2014/06/28/the-world-was-533-years-of-oil-left/11528999/}}.  Derfor er det nødvendigt at ændre på, hvilke ressourcer, der anvendes til energiproduktion, for ellers løber verden tør. Der er generel konsensus på dette, der bliver indgået aftaler om nedskæring af anvendelse af fossile brændsler og CO2 udledning, et eksempel på dette er Parisaftalen, som er en international aftale inden for FN’s klimakonvention UNFCCC \footnote{\url{https://da.wikipedia.org/wiki/Parisaftalen_(2015)}}. I Danmark er der også en grøn omstillingsplan, som har til mål at reducere den danske udledning af drivhusgasser med 70 procent i 2030\footnote{\url{https://www.gate21.dk/groen-omstilling-i-2030/}}. Danmark er i forvejen et meget grønt land, hvis man kigger på det nuværende danske elforbrug, så stammer 49 procent fra sol- og vindkraft, hvis energi produceret med biomasse inkluderes, så stammer hele 72 procent af Danmarks elforbrug fra CO2-frie energikilder \footnote{\url{https://energiwatch.dk/Energinyt/Politik___Markeder/article11839395.ece}}.  
	
	I dette projekt arbejdes med, hvordan man kan forebygge dette altoverskyggende problem, global opvarmning, som med sig bringer mange konsekvenser. Blandt andet, på baggrund af det nuværende forbrug på verdensplan, som primært består af energi fra naturlige ressourcer, så vil de naturlige ressourcer være opbrugt indenfor 53 år. I og med jordens ressourcer er ved at være opbrugt, så er det nødvendigt for menneskeheden at omlægge energiproduktionen til primært at bestå, hvis ikke udelukkende, af vedvarende energikilder, som solenergi, vindenergi og bølgeenergi m.m. I denne opgave tages der afsæt i maskinen, Dish-Stirling, som er en maskine, der omdanner solens energi til el med en Stirling motor, der er koblet til en generator. (se Figur 1 for et grafisk overblik over projektets afgrænsning) 
\end{document}
