\documentclass[a4paper,12pt]{article}

\usepackage{ucs}
\usepackage[utf8]{inputenc}
\usepackage{babel}
\usepackage{fontenc}
\usepackage{graphicx}
\usepackage{amsmath}

%\usepackage(dvips){hyperref}

\date{13-10-2020}

\begin{document}

\section{Analysis of the Stirling-Cycle Engine}
\subsection{Work done by an ideal Stirling-Cycle engine}
The net work output of a Stirling-Cycle engine can be evaluated by considering the cyclic intergral of pressure with respect to volume:

\begin{equation}
    W = -\oint p \, dV
\end{equation}
This can easily be visualised as the area enclosed by the process curves on the pressure-volume diagram. \newline
To evaluate the integral we need only consider the work done during the isothermal expansion and compression processes, since there is no work done during the isochoric processes, i.e

\begin{equation}
    W = - \left[ \int_{V_1}^{V_2} p \, dV + \int_{V_3}^{V_4} p \,\right]
    \label{eq:1}
\end{equation}
By considering the equation of state:

\begin{equation}
    pV=mRT
\end{equation}
and noting that \(T\) is constant for an isothermal process, and \(m\) is constant for a closed cycle, then an expression for work done during an isothermal process can be formulated:

\begin{equation}
    \int_{V_A}^{V_B} p \, dV = \int_{V_A}^{V_B} \frac{mRT}{V} \, dV = mRT(\ln V)_{V_A}^{V_B} = mRT \ln\left(\frac{V_B}{V_A}\right)
    \label{eq:workint}
\end{equation}
so that by substituition of Equation \eqref{eq:workint} into Equation \eqref{eq:1} we can evaluate the work integral

\begin{equation}
    W = -\left[mRT_H\ln\left(\frac{V_2}{V_1}\right)+mRT_L\ln\left(\frac{V_4}{V_3}\right)\right]
\end{equation}
where the subscipts \(H\) and \(L\) denote the high and low temperature isotherms respectively. \newline
This equation can then be further simplified by noting that \(V_4 = V_1\) and \(V_3 = V_2\) so that a final equation for work can be obtained:

\begin{equation}
    W = -mR\ln\left(\frac{V_2}{V_1}\right)\cdot (T_H - T_L)
\end{equation}


 
\end{document}
