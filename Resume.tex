\documentclass[SRC.tex]{subfiles}

\begin{abstract}
	Opgaven undersøger ved hjælp af den naturvidenskabelige metode
	om en EuroDish kan være et realistisk bud på 
	et grønnere alternativ til nuværende kraftvarmemaskiner, som 
	anvender fossile brændsler. Der redegøres for de forskellige 
	termodynamiske processor samt kredsprocessor og funktionaliteten 
	og delene af en EuroDish. I undersøgelsen og analysen opstilles og 
	udledes en matematisk model på baggrund af redegørelsen med afsæt
	i den matematiske metode, moddeleringskompetencen. Gennem analysens
	beregninger med afsæt i indsamlet empirisk materialer, bestemmes 
	den teoretiske nyttevirkning for den opstillede model til \SI{58}{\percent}.
	Den teoretiske nyttevirkning sammenlignes med nyttevirkningen af et 
	mono-krystallinsk solpanel, hvor modellens nyttevirkning er omtrent
	dobbelt så effektiv, idet solpanelets nyttevirkning er 22-\SI{27}{\percent}.
	Gennem diskussionen tydeliggøres det, at den opstillede matemtiske model 
	ikke er realistisk eller repræsentativ, da den reelle nyttevirkning for 
	en EuroDish er langt lavere og ligger på \SI{22}{\percent}. Ligeledes
	udelades flere vigtige faktorer, såsom klima, anvendelse af strøm og
	den totale afgivelse af varme til omgivelserne. Gennem de tre taksonomiske 
	niveauer har opgaven nået konklusionen, at modellen ikke kan anvendes som 
	et realistisk bud på et grønnere alternativ, da den afviger for meget fra 
	den reelle EuroDish, dog fungerer den reelle EuroDish som en god grøn 
	energiproduktion, da den primært kører på den vedvarende solenergi. 
\end{abstract}

\newpage